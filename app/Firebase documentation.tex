\documentclass{article}
\usepackage[a4paper, left=20mm, right=10mm, top=20mm, bottom=20mm]{geometry}
\usepackage[utf8]{inputenc}
\usepackage[russian]{babel}
\usepackage{amsmath}
\usepackage{fancyhdr}

\title{Документация по использованию\\Firebase Console}
\date{}
\begin{document}

\maketitle

\section{Введение}
    \paragraph*{}
    {
        Для взаимодействия с Firebase нужно перейти по ссылке \underline{console.firebase.google.com} и выбрать проект \textbf{pFood}\\
    }
\section{Добавление пользователю статуса повара/курьера}
  \paragraph{1)}
   {
        Зайдя в консоль и выбрав проект, перейти на панели слева к пункту \textbf{Authentification}
   }
   \paragraph{2)}
   {
        Во вкладке \textbf{Sing-in method}, где представлены все включенные методы аутентификации нажать на пункт \textbf{Телефон}
   }
   \paragraph{4)}
   {
        Нажать на меню \textbf{Номера телефонов для тестирования}
   }
   \paragraph{5)}
   {
        В поле \textbf{Номер телефона} набрать телефон нового повара/курьера, в поле \textbf{Код подтверждения} набрать код нового повара/курьера.
   }
   \paragraph{6)}
   {
        Зайти в приложение и войти в аккаунт, используя телефон и код, указанные шагом выше (несмотря на то что, приложение предложит ввести код из смс, вводить надо именно \textbf{код из предыдущего шага}. В последствии вход в данный аккаунт осуществляется таким же образом.)
   }
  \paragraph{7)}
   {
        Зайдя в консоль и выбрав проект, перейти на панели слева к пункту \textbf{Authentification}
   }
  \paragraph{8)}
   {
        Во вкладке \textbf{Users}, где представлены все зарегестрированные пользователи с их уникальными идентификаторами, найти нужного пользователя по номеру телефона и скопировать его идентификатор
   }
   \paragraph{9)}
   {
        На панели слева выбрать \textbf{Realtime Database}
   }
   \paragraph{10)}
   {
        Во вкладке \textbf{Данные}, где представлены все данные, с которыми приложение взаимодействует в реальном времени, нажать на белый + в черном квадрате слева от пункта \textbf{chef\_ids} или \textbf{courier\_ids} для добавления повара или курьера соответственно
   }
   \paragraph{11)}
   {
        Нажать на + справа от узла \textbf{chef\_ids} или \textbf{courier\_ids}
   }

   \paragraph{12)}
   {
        В \textbf{Названии} указать номер, следующий за последним в выпавшем списке при выполнении \\ пункта \textbf{10)}
   }

   \paragraph{13)}
   {
        В \textbf{Значение} вставить скопированный в пункте \textbf{8)} уникальный идентификатор
   }

   \paragraph{14)}
   {
        При следующей аутентификации в приложении, пользователь получит статус повара/курьера
   }

\section{Изменение времени доставки, стоимости доставки, стоимости заказа, с которой доставка бесплатна и контактных номеров}
  \paragraph{1)}
   {
        Зайдя в консоль и выбрав проект, перейти на панели слева к пункту \textbf{Realtime Database}
   }
  \paragraph{2)}
   {
        Во вкладке \textbf{Данные}, где представлены все данные, с которыми приложение взаимодействует в реальном времени, нажать на белый + в черном квадрате слева от пункта \textbf{delivery\_settings}
   }
  \paragraph{2.1)}
   {
        Для изменения времени, до которого можно сделать заказ/с которого можно сделать заказ, в выпавшем списке поменять значение поля \textbf{latest\_time}/\textbf{earliest\_time} в формате \textit{ЧЧММ}}
   \paragraph{2.2)}
   {
        Для изменения стоимости доставки в выпавшем списке поменять значение поля \textbf{delivery\_cost} (указать в рублях)
   }
   \paragraph{2.3)}
   {
        Для изменения стоимости заказа, с которой доставка бесплатна, в выпавшем списке поменять значение поля \textbf{free\_delivery\_cost}  (указать в рублях)
    }
   \paragraph{2.4)}
   {
        Для изменения контактных номеров в выпавшем списке поменять значение поля \textbf{free\_delivery\_cost} в формате \textit{номер телефона $\backslash n$ номер телефона $\backslash n$...}
   }
\section{Добавление нового блюда, новой категории}
  \paragraph{1)}
   {
        Зайдя в консоль и выбрав проект, перейти на панели слева к пункту \textbf{Realtime Database}
   }
  \paragraph{2)}
   {
        Во вкладке \textbf{Данные}, где представлены все данные, с которыми приложение взаимодействует в реальном времени, нажать на белый + в черном квадрате слева от пункта \textbf{categories}
   }
  \paragraph{2.1)}
   {
        Для добавления нового блюда в одной из категорий из выпавшего списка нажать на белый + в черном квадрате слева от названия категории
        }
   \paragraph{2.2)}
   {
       Нажать на + справа от названия категории
   }

   \paragraph{2.3)}
   {
        В \textbf{Названии} указать номер, следующий за последним в выпавшем списке при выполнении пункта \textbf{2.1)}
   }

   \paragraph{2.4)}
   {
        Не указывая ничего в \textbf{Значении}, нажать на + справа для добавления полей \textbf{description}, \textbf{imageUrl}, \textbf{name}, \textbf{price}, \textbf{products} и опционального поля \textbf{sale};
   }
   \paragraph{2.5)}
   {
  В \textbf{Значении} поля \textbf{description} указать описание блюда, в \textbf{imageUrl} - ссылку* на изображение, \textbf{name} - название, \textbf{price} - цену, \textbf{products} - продуктовый состав, в \textbf{опциональном} поле \textbf{sale} - true, если на блюдо действует скидка.
  \\ *Ссылку на изображение можно получить, загрузив соответсвующее изображение в Firebase Storage:
  \\ \quad \textbf{a)} {На панели слева выбрать пункт \textbf{Storage}}
  \\ \quad \textbf{b)} {Во вкладке \textbf{Files} выбрать/создать подходящую для изображения папку}
  \\ \quad \textbf{с)} {Загрузить изображение (изображение должно быть квадратным. Для сохранения единообразия рекомендуется изображение в формате png 500x500) и, выбрав его, в меню справа скопировать ссылку (в названии файла находится гиперссылка)}
   }

   \paragraph{3)}
   {
        Для добавление новой категории нажать на + справа от \textbf{categories}
   }
   \paragraph{3.1)}
   {
        В \textbf{Названии} указать название новой категории
   }
   \paragraph{3.2)}
   {
        Не указывая ничего в \textbf{Значении}, нажать на + справа и добавить блюда в категорию
   }

\section{Отслеживание заказов}
   \paragraph{1)}
   {
        В полях \textbf{cook}/\textbf{courier}, в подпунктах конкретного заказа в \textbf{orders}, хранятся id пользователей, которые обработали данный заказ\\
        В подпунктах поля \textbf{time} указано время, в которое заказ прошел ту или иную стадию обработки:
        \\\quad \textbf{orderTime} - время, в которое пользователь сделал заказ
        \\\quad \textbf{completeTime} - время, в которое повар выполнил/отменил заказ
        \\\quad \textbf{pickedUpTime} - время, в которое курьер забрал заказ
        \\\quad \textbf{deliveredTime} - время, в которое заказ был доставлен
   }
   \paragraph{2)}
   {Поля базы, в которых содержатся id пользователей обрабатовавших заказ, и информация обо всех заказах, прошедших определенный этап обработки (если поле остутсвует - ни один заказ не прошел данный этап)
        \\\quad \textbf{cooked} - приготовленные заказы
        \\\quad \textbf{cancelled} - отмененные заказы
        \\\quad \textbf{pickedUp} - заказы, принятые курьером
        \\\quad \textbf{delivered} - доставленные заказы
   }


\end{document} 